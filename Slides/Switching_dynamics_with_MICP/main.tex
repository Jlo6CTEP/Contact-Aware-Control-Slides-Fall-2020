\documentclass{beamer}

\pdfmapfile{+sansmathaccent.map}


\mode<presentation>
{
  \usetheme{Warsaw} % or try Darmstadt, Madrid, Warsaw, Rochester, CambridgeUS, ...
  \usecolortheme{seahorse} % or try seahorse, beaver, crane, wolverine, ...
  \usefonttheme{serif}  % or try serif, structurebold, ...
  \setbeamertemplate{navigation symbols}{}
  \setbeamertemplate{caption}[numbered]
} 


%%%%%%%%%%%%%%%%%%%%%%%%%%%%
% itemize settings

\definecolor{mypink}{RGB}{255, 150, 150}
\definecolor{myblue}{RGB}{150, 150, 255}
\definecolor{mygray}{gray}{0.8}

\setbeamertemplate{itemize items}[default]

\setbeamertemplate{itemize item}{\color{mypink}$\blacksquare$}
\setbeamertemplate{itemize subitem}{\color{myblue}$\blacktriangleright$}
\setbeamertemplate{itemize subsubitem}{\color{mygray}$\blacksquare$}

%%%%%%%%%%%%%%%%%%%%%%%%%%%%
% block settings

\setbeamercolor{block title}{bg=red!30,fg=black}


%%%%%%%%%%%%%%%%%%%%%%%%%%%%
% URL settings
\hypersetup{
    colorlinks=true,
    linkcolor=blue,
    filecolor=blue,      
    urlcolor=blue,
}

%%%%%%%%%%%%%%%%%%%%%%%%%%

\renewcommand{\familydefault}{\rmdefault}

\usepackage{amsmath}
\usepackage{mathtools}

\DeclareMathOperator*{\argmin}{arg\,min}

\usepackage{subcaption}




%%%%%%%%%%%%%%%%%%%%%%%%%%%%
% code settings

\usepackage{listings}
\usepackage{color}
\definecolor{mygreen}{rgb}{0,0.6,0}
\definecolor{mygray}{rgb}{0.5,0.5,0.5}
\definecolor{mymauve}{rgb}{0.58,0,0.82}
\lstset{ 
  backgroundcolor=\color{white},   % choose the background color; you must add \usepackage{color} or \usepackage{xcolor}; should come as last argument
  basicstyle=\footnotesize,        % the size of the fonts that are used for the code
  breakatwhitespace=false,         % sets if automatic breaks should only happen at whitespace
  breaklines=true,                 % sets automatic line breaking
  captionpos=b,                    % sets the caption-position to bottom
  commentstyle=\color{mygreen},    % comment style
  deletekeywords={...},            % if you want to delete keywords from the given language
  escapeinside={\%*}{*)},          % if you want to add LaTeX within your code
  extendedchars=true,              % lets you use non-ASCII characters; for 8-bits encodings only, does not work with UTF-8
  firstnumber=0000,                % start line enumeration with line 0000
  frame=single,	                   % adds a frame around the code
  keepspaces=true,                 % keeps spaces in text, useful for keeping indentation of code (possibly needs columns=flexible)
  keywordstyle=\color{blue},       % keyword style
  language=Octave,                 % the language of the code
  morekeywords={*,...},            % if you want to add more keywords to the set
  numbers=left,                    % where to put the line-numbers; possible values are (none, left, right)
  numbersep=5pt,                   % how far the line-numbers are from the code
  numberstyle=\tiny\color{mygray}, % the style that is used for the line-numbers
  rulecolor=\color{black},         % if not set, the frame-color may be changed on line-breaks within not-black text (e.g. comments (green here))
  showspaces=false,                % show spaces everywhere adding particular underscores; it overrides 'showstringspaces'
  showstringspaces=false,          % underline spaces within strings only
  showtabs=false,                  % show tabs within strings adding particular underscores
  stepnumber=2,                    % the step between two line-numbers. If it's 1, each line will be numbered
  stringstyle=\color{mymauve},     % string literal style
  tabsize=2,	                   % sets default tabsize to 2 spaces
  title=\lstname                   % show the filename of files included with \lstinputlisting; also try caption instead of title
}

%%%%%%%%%%%%%%%%%%%%%%%%%%%%
% tikz settings

\usepackage{tikz}
\tikzset{every picture/.style={line width=0.75pt}}


\title{Switching dynamics with MICP}
\subtitle{Contact-aware Control, Lecture 14}
\author{by Sergei Savin}
\centering
\date{Fall 2020}



\begin{document}
\maketitle


\begin{frame}{Content}

\begin{itemize}
\item Switching between LTIs
\item Mixed Integer Quadratic Programming (MIQP)
\item Big-M method: example
\item Switching between LTIs via MICP
\item Switching between nonlinear dynamics via MICP
% \item Read more
\end{itemize}

\end{frame}




\begin{frame}{Switching between different contact scenarios}
% \framesubtitle{O}
\begin{flushleft}

So far we know two major ways of describing systems with constraints:

\begin{itemize}
    \item Using explicit constraints, such as $\begin{cases}
    \bo{H} \ddot{\bo{q}} + \bo{c} = \bo{T}\bo{u} + \bo{F}^\top \lambda \\
    \bo{F} \ddot{\bo{q}} + \dot{\bo{F}} \dot{\bo{q}} = 0
\end{cases}$
    
    \item Using implicit constraints (including representation in new coordinates via projection onto the constraint manifold), such as $\mathbf{x}_{i+1} = \mathbf{A} \mathbf{x}_i + \mathbf{B} \mathbf{u}_i + \mathbf{c}$
\end{itemize}

In the first case, a new contact scenario is simply a new set of constraints. In the second - it is a new set of equations, possibly in different coordinates.

\end{flushleft}
\end{frame}




\begin{frame}{Switching between LTIs}
% \framesubtitle{O}
\begin{flushleft}

Consider groups of equations representing different modes of a switching dynamics:

\begin{equation}
    \mathbf{x}_{i+1} = \mathbf{A}_1 \mathbf{x}_i + \mathbf{B}_1 \mathbf{u}_i + \mathbf{c}_1
\end{equation}

and

\begin{equation}
    \mathbf{x}_{i+1} = \mathbf{A}_2 \mathbf{x}_i + \mathbf{B}_2 \mathbf{u}_i + \mathbf{c}_2
\end{equation}

where $\mathbf{A}_i$ is the state matrix, $\mathbf{B}_i$ is the control matrix and $\mathbf{c}_i$ is the affine term of the affine dynamics model.

\bigskip

This may represent two contact scenarios, where the original dynamics is projected onto the respective constraint manifolds. Here we assume that the state coordinates are the same between different scenarios, which doesn't have to always be the case.

\end{flushleft}
\end{frame}


\begin{frame}{Switching between LTIs}
% \framesubtitle{O}
\begin{flushleft}

This type of problem can be represented using mixed-integer approach and a big-M relaxation.

\end{flushleft}
\end{frame}


\begin{frame}{Reminder: Quadratic programming}
% \framesubtitle{Part 1}
\begin{flushleft}

General form of a quadratic program is given below:

%
\begin{equation}
\begin{aligned}
& \underset{\mathbf{x}}{\text{minimize}}
& & \mathbf{x}^\top \mathbf{H} \mathbf{x} + \mathbf{f}^\top\mathbf{x}, \\
& \text{subject to}
& & \begin{cases}
    \mathbf{A}\mathbf{x} \leq \mathbf{b}, \\
    \mathbf{C}\mathbf{x} = \mathbf{d}.
    \end{cases}
\end{aligned}
\end{equation}

where $\mathbf{H}$ is positive-definite and $\mathbf{A}\mathbf{x} \leq \mathbf{b}$ describe a \emph{convex region}.

\end{flushleft}
\end{frame}



\begin{frame}{Mixed Integer Quadratic Programming (MIQP)}
\framesubtitle{General form}
\begin{flushleft}

A general form of a mixed-integer quadratic program is:

%
\begin{equation} \label{QP}
\begin{aligned}
& \underset{\mathbf{x}, \mathbf{y}}{\text{minimize}}
& & 
\begin{bmatrix}
\mathbf{x} & \mathbf{y}
\end{bmatrix}
\begin{bmatrix}
\mathbf{H}_{11} & \mathbf{H}_{12} \\
\mathbf{H}_{21} & \mathbf{H}_{22}
\end{bmatrix} 
\begin{bmatrix}
\mathbf{x} \\
\mathbf{y}
\end{bmatrix} +
\begin{bmatrix}
\mathbf{f}_1^\top & \mathbf{f}_2\top
\end{bmatrix}
\begin{bmatrix}
\mathbf{x} \\
\mathbf{y}
\end{bmatrix}, \\
& \text{subject to}
& & \begin{cases} 
\begin{bmatrix}
\mathbf{A}_{11} & \mathbf{A}_{12} \\
\mathbf{A}_{21} & \mathbf{A}_{22}
\end{bmatrix} 
\begin{bmatrix}
\mathbf{x} \\
\mathbf{y}
\end{bmatrix}
\leq 
\begin{bmatrix}
\mathbf{b}_1 \\
\mathbf{b}_2
\end{bmatrix}, \\ 
\\
\begin{bmatrix}
\mathbf{C}_{11} & \mathbf{C}_{12} \\
\mathbf{C}_{21} & \mathbf{C}_{22}
\end{bmatrix} 
\begin{bmatrix}
\mathbf{x} \\
\mathbf{y}
\end{bmatrix} = 
\begin{bmatrix}
\mathbf{d}_1 \\
\mathbf{d}_2
\end{bmatrix},  \\
\mathbf{y} \in \mathbb{N}^n.
\end{cases}
%
\end{aligned}
\end{equation}
 
Other mixed-integer convex programs can be constructed likewise  from their pure convex counterparts.
 
\end{flushleft}
\end{frame}



\begin{frame}{Big-M method: example}
% \framesubtitle{}
\begin{flushleft}

Example: you have three systems of inequalities $\mathbf{A}_1 \mathbf{x} \leq \mathbf{b}_1$, $\mathbf{A}_2 \mathbf{x} \leq \mathbf{b}_2$, $\mathbf{A}_3 \mathbf{x} \leq \mathbf{b}_3$ and are happy if at least one holds. Then, you can write a big-M relaxation for the problem as follows:

\begin{equation}
    \begin{cases}
    \mathbf{A}_1 \mathbf{x} \leq \mathbf{b}_1 + M \cdot \mathbf{1} \cdot (1 - c_1) \\
    \mathbf{A}_2 \mathbf{x} \leq \mathbf{b}_2 + M \cdot \mathbf{1} \cdot (1 - c_2) \\
    \mathbf{A}_3 \mathbf{x} \leq \mathbf{b}_3 + M \cdot \mathbf{1} \cdot (1 - c_3) \\
    c_1 + c_2 + c_3 = 1 \\
    c_i  \in \{0, \ 1 \}
    \end{cases}
\end{equation}

where $\mathbf{1}$ is a vector of all ones. Notice that constraint $c_1 + c_2 + c_3 = 1$ can be replaced with $c_1 + c_2 + c_3 >= 1$, allowing to relax either none  or one or two regions.

\end{flushleft}
\end{frame}





\begin{frame}{Switching between LTIs via MICP}
% \framesubtitle{O}
\begin{flushleft}

With the methods described above, we can represent switching dynamics via MICP using big-M and a norm trick:


\begin{equation}
    \begin{cases}
    || -\mathbf{x}_{i+1} + \mathbf{A}_1 \mathbf{x}_i + \mathbf{B}_1 \mathbf{u}_i + \mathbf{c}_1 || \leq 
    M \cdot (1 - c_1) \\
    || -\mathbf{x}_{i+1} + \mathbf{A}_2 \mathbf{x}_i + \mathbf{B}_2 \mathbf{u}_i + \mathbf{c}_2 || \leq 
    M \cdot (1 - c_2) \\
    c_1 + c_2 = 1 \\
    c_i  \in \{0, \ 1 \}
    \end{cases}
\end{equation}


\end{flushleft}
\end{frame}




\begin{frame}{Switching between nonlinear dynamics via MICP}
% \framesubtitle{O}
\begin{flushleft}

Same way, we can represent switching nonlinear dynamics via MICP:

\begin{equation}
    \begin{cases}
    \mathbf{H}\ddot{\mathbf{q}} + \mathbf{C}\dot{\mathbf{q}} + \mathbf{g} = \mathbf{T}\mathbf{u} + \sum \mathbf{F}_i^\top \lambda_i \\
    || \lambda_i || \leq 
    M \cdot c_i \\
    || \mathbf{F}_i \ddot{\mathbf{q}} + \dot{\mathbf{F}}_i \dot{\mathbf{q}} || \leq 
    M \cdot (1 - c_i) \\
    \sum c_i = 1 \\
    c_i  \in \{0, \ 1 \}
    \end{cases}
\end{equation}


\end{flushleft}
\end{frame}



\begin{frame}{Replacing norm with linear inequalities}
% \framesubtitle{O}
\begin{flushleft}

Notice that the norm trick makes the problem into a mixed-integer SOCP.

\begin{equation}
    || \bo{A} \bo{x} + \bo{b}  || \leq 
    M \cdot (1 - c)
\end{equation}

instead we can replace the sphere described with a norm with cube described via linear inequalities:

\begin{equation}
    \bo{C} (\bo{A} \bo{x} + \bo{b}) \leq 
    M \cdot (1 - c) \bo{1}
\end{equation}

where $\bo{C} = \begin{bmatrix} \bo{I} \\ -\bo{I} \end{bmatrix}$

\end{flushleft}
\end{frame}





\begin{frame}{Thank you!}
\centerline{Lecture slides are available via Moodle.}
\bigskip
\centerline{You can help improve these slides at:}
\centerline{\href{https://github.com/SergeiSa/Contact-Aware-Control-Slides-Fall-2020}{github.com/SergeiSa/Contact-Aware-Control-Slides-Fall-2020}}
\bigskip
\centerline{Check Moodle for additional links, videos, textbook suggestions.}
\end{frame}

\end{document}
